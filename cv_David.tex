%%%%%%%%%%%%%%%%%%%%%%%%%%%%%%%%%%%%%%%%%
% "ModernCV" CV and Cover Letter
% LaTeX Template
% Version 1.11 (19/6/14)
%
% This template has been downloaded from:
% http://www.LaTeXTemplates.com
%
% Original author:
% Xavier Danaux (xdanaux@gmail.com)
%
% License:
% CC BY-NC-SA 3.0 (http://creativecommons.org/licenses/by-nc-sa/3.0/)
%
% Important note:
% This template requires the moderncv.cls and .sty files to be in the same 
% directory as this .tex file. These files provide the resume style and themes 
% used for structuring the document.
%
%%%%%%%%%%%%%%%%%%%%%%%%%%%%%%%%%%%%%%%%%

%----------------------------------------------------------------------------------------
%	PACKAGES AND OTHER DOCUMENT CONFIGURATIONS
%----------------------------------------------------------------------------------------
\documentclass[11pt,a4paper,sans]{moderncv} % Font sizes: 10, 11, or 12; paper sizes: a4paper, letterpaper, a5paper, legalpaper, executivepaper or landscape; font families: sans or roman

\moderncvstyle{casual} % CV theme - options include: 'casual' (default), 'classic', 'oldstyle' and 'banking'
\moderncvcolor{blue} % CV color - options include: 'blue' (default), 'orange', 'green', 'red', 'purple', 'grey' and 'black'

%\usepackage{lipsum} % Used for inserting dummy 'Lorem ipsum' text into the template

\usepackage[scale=0.75]{geometry} % Reduce document margins
%\setlength{\hintscolumnwidth}{3cm} % Uncomment to change the width of the dates column
%\setlength{\makecvtitlenamewidth}{10cm} % For the 'classic' style, uncomment to adjust the width of the space allocated to your name


%----------------------------------------------------------------------------------------
%	NAME AND CONTACT INFORMATION SECTION
%----------------------------------------------------------------------------------------

\firstname{David} % Your first name
\familyname{Karibe} % Your last name

% All information in this block is optional, comment out any lines you don't need
\title{Curriculum Vitae}
%\address{Mountain View}{Nairobi, Kangemi}
%\mobile{+254725247374}
\mobile{+923111680084}
%\phone{+254774433730}
%\fax{(000) 111 1113}
\email{karfes@gmail.com}
\homepage{karibe.co.ke}{} % The first argument is the url for the clickable link, the second argument is the url displayed in the template - this allows special characters to be displayed such as the tilde in this example
%\extrainfo{additional information}
\photo[70pt][0.4pt]{pictures/Photo.jpg} % The first bracket is the picture height, the second is the thickness of the frame around the picture (0pt for no frame)
\quote{"People who are really serious about software should make their own hardware" -Alan Kay}
%\quote{"Rank does not confer privilege or give power. It imposes responsibility" -Peter F. Drucker}
%----------------------------------------------------------------------------------------

\begin{document}

\makecvtitle % Print the CV title

%---------------------------------------------------------------------------------------
%   CONTACTS SECTION
%---------------------------------------------------------------------------------------
%\section{Contacts}
%\cvitem{Phone}{0725247374}
%\cvitem{Email}{karfes@uonbi.ac.ke}

%----------------------------------------------------------------------------------------
%	EDUCATION SECTION
%----------------------------------------------------------------------------------------

\section{Education}

%\cventry{Current}{Masters of Science}{Physics}{The University of Nairobi}{\textit{On-going}}{Microprocessor instrumentation and applied laser Physics}  % Arguments not required can be left empty
\cventry{2005--2009}{Bachelor of Science}{Physics (Electronics)}{The University of Nairobi}{\textit{Second Class Honours, Upper Division}}{Specialized in Applied Microprocessor Technologies and Instrumentation}

%----------------------------------------------------------------------------------------
%	WORK EXPERIENCE SECTION
%----------------------------------------------------------------------------------------
\section{Experience}

%------------------------------------------------
\subsection{Consulting}

%\cventry{2016--current}{Lead Technical Specialist}{\textsc{SunFarming (K) LTD}}{}{}{Food and Energy programme trainer(Kenya), Preliminary feasibility studies on large Photo-voltaic(PV) installation sites including GIS mapping, PV systems Design, sizing and projections in PV Sol software, PV systems remote monitoring technologies, General ICT consulting}
\cventry{2018--current}{Hardware/Firmware Engineer}{Justus Technologies}{}{}{Design and implementation of solar power monitoring IoT system using NXP K64F ARM cortex M4 and KW41Z ARM Cortex M0+ wireless controllers.
\newline{}\newline{}
Responsibilities:
\begin{itemize}
\item Design of the embedded systems hardware including modeling and spice simulation, schematic capture, PCB Layout and preparation for manufacturing.
\item Hardware testing and troubleshooting using digital meters, logic analyzers and oscilloscopes
\item Firmware development in C and register level debugging.
\item Agile development process management with version control using git and trello scrum boards
\item Preparing BOMs and aiding in procurement of required components for the project
\item 3D modeling and printing of product casing
\end{itemize}
}
\cventry{2017--current}{Co-founder and Lead Firmware Engineer}{Sign-IO}{}{}{Implementation of Bluetooth Low Energy(BLE) hardware and firmware in C using NXP KW41Z chips and Android application BLE integration.
\newline{}\newline{}
Responsibilities:
\begin{itemize}
\item Capturing circuit schematics and designing PCB layouts for BLE embedded systems
\item Developing firmware for BLE embedded systems using evaluation boards
\item Building prototypes, testing and debugging firmware in the prototypes
\item Agile development process management with version control using git and trello scrum boards
\item Managing version control repositories for the development team
\item Recruiting and training junior firmware developers
\end{itemize}
}

\section{Past jobs}
\cventry{2010--2018}{Senior Technologist}{\textsc{University of Nairobi}}{Physics Department}{}{ In charge of innovation and research laboratories in the electronics and laser applications thematic sections. Developing content, reviewing and teaching practical units in embedded systems, workshop skills and computer programming.
\newline{}\newline{}
Responsibilities:
\begin{itemize}
\item Designing embedded systems hardware and developing firmware for ARM Cortex-M targets for application in various innovation projects.
\item Design and review of embedded systems and workshop skills practical laboratory courses for the department of Physics.
\item Teaching and evaluating students in embedded systems, workshop skills and computer programming courses
\item Offering technical support to researchers at the department in instrumentation and interfacing problems
\item Facilitating and training in the University of Nairobi Solar Academy short course which happens at least twice every year.
\item Facilitating and training in the University of Nairobi Electronics Academy short course for corporate at the Physics Department
\item Repair, maintenance and development of new research set-ups
\item Procurement of new equipment (computers, embedded development Kits and assorted electronics components, microscopic imaging systems)
%\item Facilitating and attending conferences and workshops for multispectral imaging under the African Spectral Imaging Network(AFSIN)
%\item Offering technical support to undergraduate and postgraduate students in embedded systems and applied laser physics projects.
\end{itemize}
}\hspace{4em} 
\subsection{Vocational}

\cventry{2008}{August Holiday Internship}{\textsc{University of Nairobi}}{Physics Department}{Industrial Electronics Unit}{Developed an embedded system for farming automation with sensors and actuators.
\newline{}\newline{}
Detailed achievements:
\begin{itemize}
\item Learned and exercised problem solving techniques; brainstorming, problem definition and problem  disintegration.
\item Circuits design, PCB prototyping and hardware debugging
\item Applied Micro-controller programming in C
\item Project Documentation
%\item Java programming and implementing a relational database in MySql
\end{itemize}}


%------------------------------------------------

%\subsection{Miscellaneous}
%\cventry{2009--2010}{Web Developer}{IMAGINE BRANDS}{Nairobi}{}{Developing dynamic web-interfaces in PHP/MYSQL and implementing blogs using WordPress, as well as general computer software and hardware maintenance}

%----------------------------------------------------------------------------------------
%	AWARDS SECTION
%----------------------------------------------------------------------------------------

%\section{Awards}

%\cvitem{2011}{School of Business Postgraduate Scholarship}
%\cvitem{2010}{Top Achiever Award -- Commerce}

%----------------------------------------------------------------------------------------
%	COMPUTER SKILLS SECTION
%----------------------------------------------------------------------------------------

\section{Computer skills}

\cvitem{Novice}{PERL}
\cvitem{Intermediate}{\textsc{ASSEMBLY}, \textsc{Java}, Android platform, \textsc{HTML/CSS/JavaScript}, Linux admin (Kernel and Userspace system configuration)}
\cvitem{Advanced}{C/C++ programming, Electronics Systems Modeling and Simulation in systemC, Algorithms design in Matlab/Octave, KiCAD Schematic and PCB Layout Design, ELectronics Spice Modeling and Simulation in ngspice, Documents formating in \LaTeX, Openscad 3D modeling, \textsc{Python} Programming, GNU/Linux user: building software from source, configuration, hardware interfacing, shell scripting, Git (Version Control and collaboration)}

\section{Embedded Development Skills}

\subsection{Software}

\cvitem{Toolchains}{ARM GNU toolchain: General set-up and configuration, building and debugging firmware. Build automation tools: Writing Makefiles to automate firmware build process}
\cvitem{Debug Severs}{OpenOCD, USBDM, Jlink and PE-micro: experience of use for debugging ARM targets including Freescale KL Series and ST Microelectronics' microcontrollers}
\cvitem{IDEs}{Eclipse (CDT and JDT) ,NXP Kinetis Design Studio, NXP MCUXpresso with Kinetis SDK and Android Studio.}
\cvitem{Simulation}{SystemC, ngspice and Matlab: Used in teaching and research work in post-graduate studies.}
\cvitem{Wireless}{Nordic NRF24L01+ ISM wireless platform, NxP KW41Z Bluetooth Low Energy Platform paired with an android application.}
\cvitem{Interfacing}{Research equipment interface to Linux OS including serial ports,usb  spectrometers and digital cameras for Matlab platform data acquisition.}
\cvitem{KiCAD EDA}{Designing Schematics, Spice simulation, PCB layout routing and exporting PCB artwork in Gerber files for manufacturing.}

\subsection{Hardware}

\cvitem{Prototyping}{PCB prototyping using toner transfer method (double sided boards with rivetted vias) and ordering from professional services \href{https://oshpark.com}{OSH Park} and \href{https://jlcpcb.com}{JLCPCB}}
\cvitem{Testing}{PCB signal troubleshooting using digital meters, oscilloscopes and logic analyzers}
\cvitem{Debugging}{JTAG and Serial Wire Debug( SWD with ARM CMSIS-DAP)}
\cvitem{Wireless}{signal tracing using RF spectrum analyzer (SPECTRAN HF-6060 V4) for antenae testing}

\section{Workshop Skills}

\cvitem{3D Modelling}{OpenSCAD; precision 3D modelling for 3D printing and  visualization models for electronics components to be used in PCB layout 3D inspection}{ }
\cvitem{3D Printing}{ Using Ultimaker2 printer with Cura slicing tool: For product prototyping; plastic housing for embedded systems boards, optic bench alignment custom and replacement components, new setups for holographic imaging and drone parts for a remote sensing project.}


\section{Technical Training}

%\subsection{Multispectral Imaging and Remote Sensing}
%\cvitem{Remote Sensing}{Using lasers and telescopes in remote sensing, data analysis in Matlab, 2 Weeks at ICIPE-DUDUVILLE CAMPUS, Kenya (2011)}
%\cvitem{Multispectral Imaging Course }{Digital imaging labs, Flourescence imaging labs, MRI imaging, Image Processing in Maltlab, 3 months at Lund University, Sweden in 2012}
%\cvitem{Multifunctional Bench Microscope}{Design of a multispectral bench microscope and Matlab software implementation, 2 weeks at Sally Mbuor, Senegal (2015)}
%\cvitem{Optical Simulation}{Matlab modeling and simulation of optical microscopic imaging systems, 2 weeks at University of Ouaga, Burkina Faso, (2017)}

\subsection{Photovoltaic Systems}
\cvitem{Photovoltaics Trainer}{PV systems sizing and installation, 2 days at the Physics Department, University of Nairobi}
\cvitem{Tailor-made Training in Photovoltaics}{PV labs, Grid-tie systems, Roof installation, Special flexible materials with applications on portable devices, 2 weeks at Delf University, Netherlands}
\cvitem{Food and Energy trainer}{SUNfarming GmbH organized Food and Energy mixed PV systems and green house farming - Systems Design and Implementation - 2 weeks training programme at the North West University of Potchefstroom, South Africa}

%----------------------------------------------------------------------------------------
%	LANGUAGES SECTION
%----------------------------------------------------------------------------------------

\section{Languages}

\cvitemwithcomment{English}{Fluent (Speaking, reading and writing)}{}
\cvitemwithcomment{Swahili}{Fluent (Speaking, reading and writing)}{}
\cvitemwithcomment{Kikuyu}{Native language}{}

%----------------------------------------------------------------------------------------
%	INTERESTS SECTION
%----------------------------------------------------------------------------------------

\section{Hobbies}

%\renewcommand{\listitemsymbol}{-~} % Changes the symbol used for lists
%\cvlistdoubleitem{Book clubs}{}
\cvlistdoubleitem{DIY and Electronics}{Blogging}
\cvlistdoubleitem{Travelling and Photography}{CAD and 3D printing}
%\cvlistdoubleitem{3D printing}{}

\section{Referees}

\cventry{Allela Roy}{Software Engineer}{Intel Corporation}{Machine Learning and IoT}{}{
\begin{itemize}
\item Phone: +254712893128
\item email:allelaroy@gmail.com
\end{itemize}
}

\subsection{}
\cventry{James Gathu}{Technology Integration Specialist}{Escuela Campo Alegre}{Venezuela}{}{
\begin{itemize}
\item Phone: +584140314601
\item email:jamesmicwe@gmail.com
\end{itemize}
}

\subsection{}
\cventry{Dr.Justus Simiyu}{Physics Lecturer}{Maasai Mara University}{Physics Department}{}{
\begin{itemize}
\item Phone: +254722630778
\item email:jsimiyu08@gmail.com
\end{itemize}
}


%----------------------------------------------------------------------------------------
%	COVER LETTER
%----------------------------------------------------------------------------------------

% To remove the cover letter, comment out this entire block

%\clearpage
%
%\recipient{HR Department}{Corporation\\123 Pleasant Lane\\12345 City, State} % Letter recipient
%\date{\today} % Letter date
%\opening{Dear Sir or Madam,} % Opening greeting
%\closing{Sincerely yours,} % Closing phrase
%\enclosure[Attached]{curriculum vit\ae{}} % List of enclosed documents

%\makelettertitle % Print letter title
%
%%\lipsum[1-3] % Dummy text
%I am a passionate embedded systems developer with immense experience writing firmware in C programming language for the ARM Cortex-M microcontrollers. I have been working with low level debugging tools like multi-meters, oscilloscopes and logic analyzers. I also have hands on experince in circuit schematic design, PCB layout, soldering, signal troubleshooting and RF antenna testing. I have worked with NxP BLE and Nrf wireless controllers to implement wireless connectivity solutions. I use agile product development process and am familiar with version control using git. I use Linux operating system for my development work and as a daily driver. I also have some experience working with FreeRTOS both on NxP Kinetis microcontrollers and a desktop simulation setup.
%I am currently open for a permanent position as a firmware engineer in a vibrant team working to bring products to market. 
%
%\makeletterclosing % Print letter signature

%----------------------------------------------------------------------------------------

\end{document}
